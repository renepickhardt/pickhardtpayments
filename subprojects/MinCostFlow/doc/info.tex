\section{Biographical Information}

\subsection*{Personal details}

\begin{tabular}{ll} 
    Name:& Eduardo Quintana Miranda\\ 
    Affiliation:& University of Trieste, Astronomical Observatory of Trieste\\
    Course:& Astrophysics \\
    Degree Program:& PhD \\ 
    Country:& Italy\\
    E-mail:& \url{eduardo.quintana@pm.me} \\ 
    Github:& \url{github.com/Lagrang3} \\
    Discord:& \texttt{eduardo-qm}
\end{tabular}

\subsection*{Educational background.}

\begin{itemize} 
    \item 2008--2013. BSc. Nuclear Physics, University of Havana,
    \item 2015--2018. MSc. Theoretical Physics, University of Trieste, 
    \item
    2018--2019. Master in High Performance Computing, International School of
    Advanced Studies (SISSA), Trieste, 
    \item 2019--present. PhD in astrophysics,
    University of Trieste.  
\end{itemize}

\subsection*{Programming background.} 
My main programming language is C++, but I am also profiecient in C and python.

I started programming back in 2009 during my 2nd year of BSc.  The main drivers
were physics simulations and programming competitions.  From 2009 until 2013
I've participated intensively in those kind of competitions, 
the most important were the
ACM-ICPC\footnote{\url{icpc.global}} competitions held every year, my team
managed to classify many times to the regional phase, which in our geographical
context where named \emph{Caribbean Finals of the ICPC}.  Until this date I
participate once in a while on \url{codeforces.com} rounds by the username
\texttt{Lagrang3}\footnote{\url{https://codeforces.com/profile/Lagrang3}}.
I have a fair knowledge of a variety of classical algorithms somewhat seasoned
by my participation on programming competitions. I am familiar with shortest
path, max-flow and linear min-cost flow graph problems and textbook algorithms
for solving them.

I've acquired some professional programming skills during the Master in High
Performance Computing. Advanced C++, Python, bash, git, and parallel programming
in OpenMP, MPI and Cuda where among the topics taught in that curriculum.

In 2021 I've succesfully completed a Google Summer of Code project for the
proposal of FFT utilities in Boost Math 
library\footnote{\url{https://github.com/BoostGSoC21/math-fft-report/releases/download/v1.1/gsoc-report.pdf}}.

Today, I am doing a PhD in astrophysics working on the developement of a
simulation code
to study the effects of general relativity in the formation of cosmological
structures.\footnote{\url{https://github.com/Lagrang3/gevolution-1.2/tree/gev-api}}
